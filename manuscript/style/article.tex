\documentclass[article]{jss}

%%%%%%%%%%%%%%%%%%%%%%%%%%%%%%
%% declarations for jss.cls %%%%%%%%%%%%%%%%%%%%%%%%%%%%%%%%%%%%%%%%%%
%%%%%%%%%%%%%%%%%%%%%%%%%%%%%%

%% almost as usual
\author{Jonathan D. Rosenblatt\\Ben Gurion University \And 
        Yakir Berchenko\\ Gertner Institute for \\ Epidemiology and \\ Health Policy Research \And
        Simon D. Frost\\ Cambridge University}
\title{The \pkg{chords} \proglang{R} Package- A Principled Approach to Respondent Driven Sampling}

%% for pretty printing and a nice hypersummary also set:
\Plainauthor{Jonathan D. Rosenblatt, Yakir Berchenko, Simon D. Frost} %% comma-separated
\Plaintitle{The chords R Package- A Principled Approach to Respondent Driven Sampling} %% without formatting
\Shorttitle{\pkg{chords} for Respondent Driven Sampling} %% a short title (if necessary)

%% an abstract and keywords
\Abstract{
  The abstract of the article.
}
\Keywords{RDS, estimation, counting-process, \proglang{R}}
\Plainkeywords{RDS, estimation, counting-process, R} %% without formatting
%% at least one keyword must be supplied

%% publication information
%% NOTE: Typically, this can be left commented and will be filled out by the technical editor
%% \Volume{50}
%% \Issue{9}
%% \Month{June}
%% \Year{2012}
%% \Submitdate{2012-06-04}
%% \Acceptdate{2012-06-04}

%% The address of (at least) one author should be given
%% in the following format:
\Address{
  Jonathan D. Rosenblatt\\
  Department of Industrial Engineering and Management\\
  Faculty of Engineering\\
  Ben Gurion University of the Negev\\
  P.O. 653, Beer Sheva, 8410501\\
  E-mail: \email{johnros@bgu.ac.il}\\
  URL: \url{http://www.john-ros.com/}
}
%% It is also possible to add a telephone and fax number
%% before the e-mail in the following format:
%% Telephone: +43/512/507-7103
%% Fax: +43/512/507-2851

%% for those who use Sweave please include the following line (with % symbols):
%% need no \usepackage{Sweave.sty}

\usepackage{amsmath, amsthm, amssymb}
\usepackage{natbib}
\usepackage{url}


% General purposes commands
\newcommand{\limprob}{\xrightarrow{P}}
\newcommand{\limdist}{\xrightarrow{D}}
\newcommand{\chords}{\textit{chords}}
\newcommand{\gauss}[1]{\mathcal{N}\left(#1\right)}
\newcommand{\naive}{na\"{\i}ve}
\newcommand{\Naive}{Na\"{\i}ve}
\newcommand{\set}[1]{\left\{ #1 \right\}}
\newcommand{\setII}[1]{\left\{ #1 \right\}} % A set
\newcommand{\rv}[1]{\mathbf{#1}} % A random variable
\newcommand{\x}{\rv x} % The random variable x 
\newcommand{\y}{\rv y} % The random variable x 
\newcommand{\U}{\rv u} % The random variable x 
\newcommand{\T}{\rv t} % The random variable x 
\newcommand{\X}{\rv X} % The random variable x 
\newcommand{\Y}{\rv Y} % The random variable y
\newcommand{\expect}[1]{\mathbf{E}\left[ #1 \right]} % The expectation operator
\newcommand{\expectg}[2]{\mathbf{E}_{\rv{#1}}\left[ \rv{#2} \right]} % An expectation w.r.t. a particular random 
\newcommand{\bigO}{\mathcal{O}}
\newcommand{\bigOprob}{\mathcal{O}_P}
\newcommand{\smallO}{o}
\newcommand{\smallOprob}{o_P}
\newcommand{\aka}{{a.k.a.\ }}
\newcommand{\Aka}{{A.k.a.\ }}

% Paper specific commands:
\newcommand{\Ns}{N_1,N_2,\dots}
\newcommand{\hatNs}{\hat{N}_1, \hat{N}_2,\dots}


%% end of declarations %%%%%%%%%%%%%%%%%%%%%%%%%%%%%%%%%%%%%%%%%%%%%%%


\begin{document}
%% include your article here, just as usual
%% Note that you should use the \pkg{}, \proglang{} and \code{} commands.

\section{Introduction}

As the name suggests, Respondent Driven Sampling (RDS) is a framework for sampling by chain-referral.
RDS is a bundle of a sampling mechanism and analysis methods, most common in the study of marginalized populations which do not lend themselves to simple sampling \citep{heckathorn_respondent-driven_1997,heckathorn_respondent-driven_2002}. 

In RDS seeds are selected -- usually by convenience -- from the target population, and given coupons. 
They use these coupons to recruit others, who themselves become recruiters. 
Recruits are given an incentive, usually money, for taking part in the survey, and also for recruiting others.
This process continues in recruitment waves until the survey is stopped, usually when a target sample size is reached.

With the above sampling mechanism, highly connected individuals will be overrepresented in the sample.
If the attribute of interest is correlated with an individual's degree, as is often the case (e.g. HIV), \naive estimates will be biased towards the state of the highly connected subgroups.
An unbiased Horowitz-Thompson-type estimator \citep{horvitz_generalization_1952} would require the knowledge of frequency of each degree. 
Clearly, the frequency of each degree will also be biased towards higher degrees, and thus cannot be recovered from the knowledge of individuals' degrees alone. 
The common remedy to this matter is the inverse-degree weighting heuristic \citep{crawford2015hidden,guntuboyina2012impossibility}.

In \cite{berchenko_modeling_2013} we proposed a generative model for RDS.
The model is based on the idea that RDS spreads like an epidemic, and we can thus borrow epidemiological generative models. In particular SIR [TODO: add citation], for likelihood based inference.
Having assumed a generative process, we can now estimate degree frequencies, introduce covariates, check the goodness of fit, and discuss the model's assumptions. 

The details of the assumed generative model can be found in \cite{berchenko_modeling_2013}, but the essentials are now detailed for completeness.

Denote $N_k$ the unknown population frequency of degree $k$, i.e., the number of individuals with $k$ ``friends''. 
Denote $N:=\sum_k N_k$, the total population size. 
Denote by $x_t$ the degree of the respondent recruited at time $t$. 
Our task is to estimate $\hat{N}_1, \hat{N}_2,\dots$, based on a sample of $x_{t_1},\dots, x_{t_\tau}$. 
Denote $\lambda_{k,t}$, the probability of recruiting an individual with degree $k$ in the time interval $[t,t+\Delta]$.
We assume the following generative multivariate counting model:
\begin{align}
\label{eq:model}
 \lambda_{k,t} =  \beta_k \frac{n_t}{N}  (N_k - n_{k,t}) \Delta t + \smallO(\Delta t),
\end{align}
where $\beta_k$ is some base recruitment rate, 
$n_t$ is the number of recruiting individuals at time $t$, 
and $n_{k,t}:=\sum_{s\leq t} I_{\set{x_s=k}}$ the number of recruited individuals of degree $k$. 
It thus follows that $\frac{I_t}{N}$ is the recruiting population proportion, and $(N_k - n_{k,t})$ the recruitment ``potential'' of degree $k$.
In these terms, our model implies that the recruitment probability, in short enough time periods, is proportional to the recruiting team (\aka the ``snowball'') and the recruiting potential. 
For more on this model, and relations to other existing models, see \citet{berchenko_modeling_2013}.

Equipped with this generative model, we may state the estimation of $\Ns$ as a likelihood maximization problem. 
Moreover, we show in \cite{berchenko_modeling_2013} that the maximum likelihood estimator (MLE) of $\Ns$ is separable for the various $k$, and independent of $\beta_k$. 
This fact is used by \citet{berchenko_modeling_2013} for proof of the asymptotic properties of $\hatNs$, and in the  \pkg{chords} package to accelerate the estimation task.

MLEs for $\Ns$ in model (\ref{eq:model}) have an interesting finite sample property: They tend to return $\hat{N}_k=\infty$ with non null probabilities. 
This clearly calls for some regularization. 
The Bayesian framework, a celebrated regularization device, will not work in this problem. 
Informally speaking, this is because any shrinkage of $\infty$ is still $\infty$. 
In our package, we have implemented several regularization devices. Some very crude heuristics, and some less so. 
In particular, we generalized the ideas of \cite{musa_estimating_1991} and \cite{osborne_inference_2000} to our multivariate case.
The former proposing a jacknife-type resampling scheme, and the latter an \emph{integrated likelihood} approach.

The usage of the package, the MLE estimator, and the small-sample modifications, is demonstrated in the next section.



\section{Work Flow}

We start by demonstrating the workflow on some data supplied with the package. 
The data consists of [TODO: describe data].
For more details, see \citet{salganik_assessing_2011}. 

At a high-level, the work-flow has the same functional-object-oriented flavour as \pkg{ggplot2}:
All the data required for the estimation is contained in the \code{rds-class} objects, and so are the results. 
Like \pkg{ggplot2}, and unlike object-oriented, the estimation does not modify the object on which it operates. The output will thus be a new object, with the same data, and a new estimate. 

A typical work-flow consists of:
(a) casting the data into the RDS file format\footnote{Defined in \url{http://www.respondentdrivensampling.org/reports/RDSAT_7.1-Manual_2012-11-25.pdf}.}, 
(b) initializing an \code{rds-class} object, 
(c) compute initial estimator.
(d) use a modified estimator to deal with infinite values. 



%\begin{CodeChunk}
%\begin{CodeInput}
%first input first line
%first input second line
%\end{CodeInput}
%\begin{CodeOutput}
%output of first input
%\end{CodeOutput}
%\begin{CodeInput}
%second input
%\end{CodeInput}
%\end{CodeChunk}




\section{Some Technicalities}


\section{Conclusion}


\section{Future Work}


\bibliographystyle{abvnat}
\bibliography{RDS}


\end{document}
